% INTRODUÇÃO-------------------------------------------------------------------

\chapter{INTRODUÇÃO}
\label{chap:introducao}

Esse trabalho de pesquisa visa esclarecer o atual cenário de licenciamento de projetos open source da linguagem de programação Javascript. Essa analise será feita levando em consideração uma população de projetos hospedados no Github.

Licença de um software é uma das partes não-executáveis mais importantes de qualquer sistema de software. As licenças de software não apenas orientam como alguém pode usar e reutilizar um determinado sistema de software, mas também têm o potencial de influenciar em aspectos de manutenção e evolução de um software. No entanto, devido à sua natureza não técnica, desenvolvedores de software geralmente tem pouco entendimento que levam a potenciais maus usos de licenças de software. Estudos anteriores relataram vários problemas relacionados a conflitos e inconsistências em licenças de software, o que poderia, inclusive, levar a erros de software. Esses problemas ocorrem devido a vários motivos, incluindo a falta de uma documentação de apoio à qual desenvolvedores de software possam se referir ao escolher uma licença ou devido à falta de ferramentas adequadas nas quais desenvolvedores possam se apoiar quando para, por exemplo, escolher entre mais de uma licença. 

Embora suporte inicial tenha sido disponibilizado recentemente, pouco ainda se sabe sobre \cite{Almeida:2017:SDU:3101414.3101416} como desenvolvedores escolhem uma licença para utilizar, \cite{Bodden:2018:SSA:3183399.3183401} o quão frequente são outros tipos de violações ou até tentativas de burlar a licença, ou \cite{Cartaxo:2016:EBT:2961111.2962603} o quão comum são projetos de software livre lançados sem uma licença. O objetivo geral desta proposta é duplo: primeiro, exploraremos melhor as necessidades, os desafios e os problemas que os desenvolvedores enfrentam quando lidam com licenças de software. Então, com um melhor entendimento desses problemas, planejamos introduzir livros de receitas, técnicas e ferramentas para melhor apoiar os desenvolvedores (por exemplo, recomendar licenças de software com base no uso do código-fonte). Esse trabalho é particularmente relevante na era em que um fluxo constante de projetos de software livre são lançados diariamente, mas pouco ou nenhum cuidado é dado às licenças de software.

\section{Objetivos}
\label{sec:objetivos}

\subsection{Objetivo geral}
\label{subsec:objetivogeral}
Analisar e conhecer o atual cenário de licenciamento de projetos open source da linguagem de programação Javascript, que estão hospedado nos repositórios do github.

\subsection{Objetivos específicos}
\label{subsec:objetivosespecificos} 

\begin{itemize}
\item Identificar qual o quantitativo de licenças de software e não-software encontradas nos projetos;
\item Identificar a quantidade de projetos com mais de uma licença;
\item Compreender como as licenças de software estão distribuídas quantitativamente;
\item Medir a proporção de licenças permissivas e restritivas encontradas nos projetos;
\item Descrever como as licenças são expressas nos projetos;
\item Distinguir as licenças reconhecidas pelas SPDX;
\end{itemize}

\section{Estrutura do trabalho}
\label{sec:estrututaTrabalho}

Este trabalho está dividido em seis seções, referências e anexos.

Na seção 1 é apresentado o contexto no qual o trabalho está inserido, a justificativa e os objetivos almejados.

Na seção 2 é apresentado a metodologia desse trabalho assim como também as técnicas e procedimentos de coleta e analise dos dados.

A fundamentação teórica sobre as temáticas relacionadas com essa pesquisa é apresentada na seção 3.

Na seção 4, os resultados são apresentados juntamente com suas devidas discussões.

Os trabalhos relacionados são apresentados na seção 5, onde é explanado os trabalhos científicos recentes que abordam temas relacionados com este trabalho. Na seção seguinte (seção 6) será apresentado as ameaças a validade do trabalho, citando todas a limitações e barreiras que esse trabalho enfrentou.

Finalizando, a seção 7 faz as devidas conclusões e apresenta sugestões para trabalhos futuros.
