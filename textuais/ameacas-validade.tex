% AMEAÇAS A VALIDADE--------------------------------------------------------

\chapter{AMEAÇAS A VALIDADE DO TRABALHO}
\label{chap:AmeacasAValidade}

%\section{Validação de Construção}
As ameaças para construir a validade do trabalho dizem respeito à relação entre teoria e observação, e relacionam-se a uma possível imprecisão da medição ao extrair dados usados neste estudo. Os resultados ficam intimamente ligados a acurácia e limitações da ferramenta utilizada para gerar os dados para analise.

%\section{Validação Interna}
%\section{Validação Externa}
Ameaças à validade externa representam a capacidade de generalizar as observações em nosso estudo. Nosso estudo quantitativo é baseado na análise de mais de 1.500 repositórios hospedados no Github da linguagem Javascript.  No entanto, a generalização de nossos resultados qualitativos é limitada a linguagem Javascript escolhida no dataset que podem não espelhar o verdadeiro cenário da população completa de repositórios hospedados no Github.
%\section{Confiabilidade}


\begin{comment}
 Na mineração dos repositórios Git, contamos com a API GitHub e a linha de comando git
utilitário. Essas são as duas ferramentas em desenvolvimento ativo e têm um apoio comunitário
eles. Além disso, a API do GitHub é a interface principal para extrair informações do projeto.
Não podemos excluir imprecisões devido à implementação dessa API. Em termos de licença
Na classificação, contamos com Ninka, uma abordagem de ponta que demonstrou ter
95% de precisão (German et al., 2010b); no entanto, nem sempre é capaz de identificar o
licença (15% do tempo nesse estudo). No que diz respeito à codificação aberta realizada em
No contexto do RQ4, identificamos, por meio de uma amostra estratificada, uma amostra de confirmação
mensagens e discussões de rastreadores de problemas grandes o suficiente para garantir um erro de ± 10% com um
nível de confiança de 95%. Essa amostra foi identificada a partir da confirmação do candidato
mensagens e discussões identificadas por meio da correspondência de padrões, usando as palavras-chave
Tabela 1. Embora tenhamos como objetivo criar um conjunto abrangente de palavras-chave relacionadas ao licenciamento, é
é possível que tenhamos perdido discussões relacionadas ao licenciamento que não correspondam a nenhuma dessas palavras-chave.

Ameaças à validade interna podem estar relacionadas a fatores de confusão internos ao nosso estudo,
isso poderia ter afetado os resultados. Para as alterações no licenciamento atômico, reduzimos a ameaça
de ter o tamanho do projeto como um fator de confusão, representando as presenças de uma alteração específica em cada confirmação. Uma alteração de licença geralmente é tratada em uma determinada instância e não
frequência. Ao usar a análise no nível de confirmação, impedimos que o número de arquivos seja inflado
os resultados para que eles não sugiram inadequadamente um grande número de alterações ocorridas
em um projeto. Para analisar as mudanças entre os projetos, adotamos uma abordagem binária de analisar a presença de um padrão. Portanto, um projeto em particular não dominaria nossos resultados
devido ao tamanho. Para limitar a subjetividade da codificação aberta, as classificações sempre foram realizadas por dois dos autores e, em seguida, todos os casos de classificação discográfica foram discutidos.
conforme explicado na Seção 3.3.

 No entanto, a generalização de nossos resultados qualitativos é limitada aos sete idiomas considerados e é
suportado pelo número relativamente baixo de sistemas considerados (ou seja, 1.160) devido ao manual
esforço necessário para identificar a lógica por trás das decisões de licenciamento (bem como
o número limitado de repositórios em potencial com mensagens de confirmação ou problemas relacionados à licença
discussões).
O crescimento exponencial e a popularidade do GitHub como forja pública indicam que ele representa
uma grande parte da comunidade de código aberto. Enquanto o crescimento exponencial ou relativo
Como a juventude dos projetos pode impactar os dados, essas duas características representam a
crescimento do desenvolvimento de código aberto e não deve ser descartado. Além disso, o GitHub
contém um grande número de repositórios, mas pode não ser necessariamente um
conjunto de todos os projetos de código aberto ou mesmo todos os projetos Java. Contudo, o grande número de
projetos em nosso conjunto de dados (e valores de métricas de diversidade relativamente altos, como mostrado na Seção 3.4)
nos dá confiança suficiente sobre as descobertas obtidas. Avaliação adicional dos projetos

\end{comment}