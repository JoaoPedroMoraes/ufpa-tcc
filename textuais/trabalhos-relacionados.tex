% REVISÃO DE LITERATURA--------------------------------------------------------

\chapter{TRABALHOS RELACIONADOS}
\label{chap:fundamentacaoTeorica}
Sera feita uma analise de trabalhos relacionados de outros autores sobre o tema de trabalho, dentro desse cenário vasto do licenciamento de software, sera abordado trabalho que tange a temática abordada nesse trabalho de conclusão.

No artigo \cite{Vendome:2018:DDW:3180155.3180221} é explanado como os erros de licenciamento de software são importantes, pois com a variedade de licencias de software (restritivas e permissivas) existentes podem ser criadas confusões na hora do licenciamento do software, que não necessariamente limitam a compatibilidade das licenças. Esse estudo é resultado de uma analise manual de 1.200 discussões relacionadas a erros de licenciamento realizadas em issue trackers e em cinco mailing lists de discussões de projeto da comunidade open source. 

Neste trabalho notou-se que os erros de licenciamento se dão por motivos diversos, pois compreendeu-se que as leis de direitos autorais são complexas para a maioria dos desenvolvedores. Foi observado também que em alguns projetos de código aberto, existe a discussão em termos de se uma ação tem o potencial de violar direitos autorais, onde geralmente essas discussões estão entre os desenvolvedores, que não necessariamente tem o entendimento adequado da lei de direitos autorais e geralmente não tem aconselhamento jurídico profissional. 

No artigo \cite{Meloca:2018:UUI:3196398.3196427} é abordada uma questão importante no cenário de licenciamento de software que é negligenciada: o uso de licenças de de código aberto, mas que não foram formalmente aprovadas pela Open Source Iniative (OSI), pois quando um desenvolvedor lança um software sob uma licenças não aprovada, mesmo que o interesse seja torna-lo de código aberto, o  autor pode não esta concedendo os direitos exigidos para aqueles que usam o software. 

Para descobrir as razões por trás desse uso forma minerados dados de 675k projetos de código aberto e suas versões, e também 76 desenvolvedores que publicaram alguns desses projetos. O estudo foi realizado e varias características foram observadas. Como que não é incomum para desenvolvedores mudarem a licenças não aprovadas para licenças aprovadas pela OSI, que quando os desenvolvedores foram questionados sobre a mudança os mesmo mencionaram que que essa transição foi devido a uma melhor compreensão das desvantagens do uso de uma licença aprovadas, mostrando mais uma vez que essa parte de licenciamento de software necessita de um maior esclarecimento de como as licencias funcionam para o desenvolvedores desses projetos.

Em cerca de 24\% dos projetos analisados, observou-se que usavam pelo menos uma licença não aprovada pela OSI, sendo que a maioria dessas licenças não aprovadas eram projetos sem licença alguma. Quando o estudo entrou em contato com os mantenedores dos projetos, 46\% dos entrevistados não levam em consideração a licença quando escolhem uma dependência de pacote e alguns entrevistados acreditam que as licenças não aprovadas são mais abertas e mais simples de usar.

No artigo \cite{Vendome:2017:LUC:3106879.3106907} é feito um extenso estudo empírico, em larga escala no gitHub, sobre alterações de licenças. Foi feita uma investigação de caráter qualitativa e quantitativa de quanto e por que os desenvolvedores adotam ou alteram as licenças dos softwares. Na analise feita nas mensagens de commits e nas discussões de issues destacou-se que as informações oferecidas com relação à escolhas/alterações do licenciamento geralmente são bastantes limitadas, com isso um desenvolvedor interessado em reutilizar o código seria forçado a verificar o código fonte para entender o licenciamento exato ou solicitar esclarecimento. Além disso, os motivos por trás da alteração geralmente não são bem documentados, mas notou-se uma tendencia em alterações de licenciamento predominantemente em direção ou entre licenças permissivas, que facilitam algum tipo de trabalho derivado e redistribuição, por exemplo, em produtos comerciais.

Foi observado também que os desenvolvedores interpretam as implicações do licenciamento de maneira diferente, o que gera mal-entendidos nos termos de reutilização, culminando em uma reutilização de código problemática para os desenvolvedores devido ao licenciamento

A falta de padronização na maneira como a documentação do licenciamento é expresso nos projetos também foi um problema observado, visto que alguns projetos apresentavam a licenças no diretório pai, outras no cabeçalho do código fonte. Assim com foi observado que o uso comercial dos projetos é uma preocupação da comunidade, mostrando assim a importância de licenças como MIT e Apache para esses fins, que aparecem com bastante frequência nos projetos open source atualmente.

No artigo \cite{BSDeMIT} foi apresentado um estudo empírico de como as famílias de licenças BSD e MIT variam de sua definição original. A família de licenças BSD se aproxima dos modelos SPDX existentes, com pouca variabilidade adicional. A família de licenças MIT foi encontrado para ser muito mais fragmentado e altamente personalizado, incluindo a criação de várias variantes especializadas baseadas na licença original do X11, personalização do texto "autores ou detentores de direitos autorais", alterações ortográficas e adição e remoção de condições, subsídios e sentenças completas. Pequenas alterações no modelo SPDX para o Licença MIT e à lista SPDX de palavras equivalentes acomodariam algumas a variação essencial encontrada na licença a um baixo custo.

Observou-se que os licenciadores alteram o texto das licenças padrão de código aberto para vários propósitos, incluindo a personalização da licença com o nome de sua organização específica, adicionando ou removendo condições e alteração de ortografia ou pontuação. Padrões de licenciamento de código aberto, como O SPDX pode ser afetado pela variabilidade nas licenças, pois a variabilidade pode alterar os requisitos legais. significado das licenças, criando problemas legais na correspondência de uma licença alterada. Em contraste, a exigência de uma "combinação perfeita" excessivamente rigorosa de licenças de código aberto para o padrão pode resultam na exclusão de muitos textos de licença com variabilidade insignificante. 